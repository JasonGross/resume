% LaTeX file for resume
% This file uses the resume document class (res.cls)

\documentclass[11pt]{res}
\usepackage[T1]{fontenc}
\usepackage{xifthen,verbatim}
\usepackage{hyperref}
\usepackage{etoolbox}[2011/01/03]


\usepackage[backend=biber,bibencoding=utf-8,maxbibnames=99,sorting=ymdnt,noerroretextools=true,style=alphabetic]{biblatex}
% http://tex.stackexchange.com/a/140641
\DeclareSortingTemplate{ymdnt}{
    \sort{
        \field{presort}
    }
    \sort[final]{
        \field{sortkey}
    }
    \sort[direction=descending]{
        \field{sortyear}
        \field{year}
        \literal{9999}
    }
    \sort[direction=descending]{
        \field[padside=left,padwidth=2,padchar=0]{month}
        \literal{99}
    }
    \sort[direction=descending]{
        \field[padside=left,padwidth=2,padchar=0]{day}
        \literal{99}
    }
    \sort{
        \field{sortname}
        \field{author}
        \field{editor}
        \field{translator}
        \field{sorttitle}
        \field{title}
    }
    \sort{
        \field{sorttitle}
        \field{title}
    }
}

\makeatletter

\let\origifempty\ifempty
\undef\ifempty
\let\origforlistloop\forlistloop
\usepackage{etextools}
\let\eifempty\ifempty
\let\ifempty\origifempty
\let\forlistloop\origforlistloop
\def\preetextools{\let\ifempty\eifempty}


\newif\ifmultitexcompile
\multitexcompiletrue


\newcommand\multitex@separator{-}
\newwrite\multitex@temp
% \immediate\openout\multitex@temp{\jobname.mtx}
\newtoks\multitex@names
\multitex@names={}
\newcommand{\multitex@postprocess}[1]{%
  \immediate\expandafter\closeout\csname @multitex@#1@file\endcsname
  \ifmultitexcompile
    \immediate\write18{pdflatex \csname @multitex@#1@name\endcsname}%
    \immediate\write18{biber \csname @multitex@#1@barename\endcsname}%
    \immediate\write18{pdflatex \csname @multitex@#1@name\endcsname}%
  \fi
}
\newcommand{\onlymultitex}[1]{\FE@ifstar{#1}{\only@other@multitex}{\only@multitex}}
\newcommand{\only@multitex}[2]{#2}
\newcommand{\only@other@multitex}[2]{}
{\catcode`\#=12
  \global\def\@pound{#}
}
\newcommand{\DeclareFile}[2][]{
  \global\multitex@names=\expandafter{\the\multitex@names,#2}
  \expandafter\xdef\csname @multitex@#2\endcsname{\xifempty{#1}{#2}{#1}}
  \expandafter\xdef\csname @multitex@#2@name\endcsname{"\jobname\multitex@separator\csname @multitex@#2\endcsname.tex"}
  \expandafter\xdef\csname @multitex@#2@auxname\endcsname{"\jobname\multitex@separator\csname @multitex@#2\endcsname.aux"}
  \expandafter\xdef\csname @multitex@#2@barename\endcsname{"\jobname\multitex@separator\csname @multitex@#2\endcsname"}
  \expandafter\newwrite\csname @multitex@#2@file\endcsname
  \immediate\expandafter\openout\csname @multitex@#2@file\expandafter\endcsname\csname @multitex@#2@name\endcsname
  \AtBeginDocument{%
    \multitex@write{#2}{%
      \detokenize{\newcommand{\onlymultitex}[1]}%
        {\detokenize{\csname FE@ifstar\endcsname}%
          {\@pound1}%
          \detokenize{%
            {\csname only@multitex\endcsname}{\csname only@multitex\endcsname}%
          }%
        }%
      \detokenize{%
        \expandafter\newcommand\csname only@multitex\endcsname[2]}%
        {%
          \noexpand\renewcommand*{\noexpand\do}[1]{%
            \noexpand\ifstrequal{\@pound\@pound1}{#2}%
              {\@pound2\noexpand\listbreak}%
              {}%
            }%
          {\noexpand\edef\noexpand\foo{\@pound1}\noexpand\expandafter}%
          \noexpand\expandafter\noexpand\docsvlist\noexpand\expandafter{\noexpand\foo}%
        }
      \detokenize{%
        \expandafter\let\csname only@other@multitex\expandafter\endcsname\csname only@multitex\endcsname
      }%
    }%
  }
  \AtEndDocument{\multitex@postprocess{#2}}
}
\newcommand{\multitex@write}[2]{\xifblank{#1}{}{\immediate\expandafter\write\csname @multitex@#1@file\endcsname{#2}}}
\edef\multitex@tempfilename{\jobname.mtx}
\newcommand{\multitex@pre@writetothis}{\immediate\openout\multitex@temp\multitex@tempfilename}%\long\gdef\multitex@thiscode{}}%
\newcommand{\multitex@writetothis}[1]{\immediate\write\multitex@temp{#1}}%\ExpandNext{\long\gappto\multitex@thiscode}{\expandnext{\relax\scantokens}{#1}}}%
\newcommand{\multitex@post@writetothis}{\immediate\closeout\multitex@temp\relax\input{\multitex@tempfilename}}%\show\multitex@thiscode\multitex@thiscode\relax}%
\newcommand{\multitex@writetoothers}[2][\the\multitex@names]{{\preetextools\ExpandNext{\forcsvloop}{#1}\do{\multitex@write{##1}{#2}}}}
\newcommand{\DeclareDocumentClass}[1]{\multitex@writetoothers{\detokenize{#1}}}
\AtBeginDocument{\multitex@writetoothers{\detokenize{\begin{document}}}}
\AtEndDocument{\multitex@writetoothers{\detokenize{\end{document}}}}
\newcommand{\includein}[2][]{\multitex@pre@writetothis\multitex@writetothis{\detokenize{#2}}\multitex@post@writetothis\multitex@writetoothers[#1]{\detokenize{#2}}}
\AfterEndEnvironment{multitex}{\multitex@post@writetothis}
\newenvironment{multitex}[1][\the\multitex@names]{%
  \@bsphack
  \begingroup% Lets Keep the Changes Local
    \multitex@pre@writetothis
    \let\do\@makeother\dospecials\catcode`\^^M\active
    \def\verbatim@processline{\multitex@writetothis{\the\verbatim@line}\multitex@writetoothers[#1]{\the\verbatim@line}}%
    \verbatim@start\space % I'm not sure why I need the space, but I seem to need it.
}{\endgroup
  \@esphack
}
% only others
\newenvironment{multitex*}[1][\the\multitex@names]{%
  \@bsphack
  \begingroup% Lets Keep the Changes Local
    \let\do\@makeother\dospecials\catcode`\^^M\active
    \def\verbatim@processline{\multitex@writetoothers[#1]{\the\verbatim@line}}%
    \verbatim@start\space % I'm not sure why I need the space, but I seem to need it.
}{\endgroup
  \@esphack
}
\makeatother

\DeclareFile{curriculum-vitae}
%\DeclareFile{mathematics}
%\DeclareFile{computer-science}
%\DeclareFile{curriculum-vitae-long}
%\DeclareFile{curriculum-vitae-mathematics}
%\DeclareFile{curriculum-vitae-computer-science}
\DeclareDocumentClass{\documentclass[11pt]{res}}

\begin{multitex}
\edef\resumefiles{mathematics,computer-science}
\edef\cvshortspecialfiles{curriculum-vitae-mathematics,curriculum-vitae-computer-science}
\edef\cvshortfiles{curriculum-vitae,\cvshortspecialfiles}
\edef\cvlongfiles{curriculum-vitae-long}
\edef\cvfiles{\cvshortfiles,\cvlongfiles}
\edef\highschoolfiles{curriculum-vitae-long}
\edef\mathfiles{mathematics,curriculum-vitae-mathematics}
\edef\cvlongandmathfiles{\mathfiles,curriculum-vitae-long}
\edef\cvandmathfiles{\cvlongandmathfiles,curriculum-vitae,curriculum-vitae-computer-science}
\edef\computersciencefiles{computer-science,curriculum-vitae-computer-science}
\edef\cvlongandcomputersciencefiles{\computersciencefiles,curriculum-vitae-long}
\edef\cvandcomputersciencefiles{curriculum-vitae,curriculum-vitae-mathematics,\cvlongandcomputersciencefiles}
\edef\allexceptcv{\resumefiles,\cvshortspecialfiles}
\edef\allexceptcvlongfiles{\resumefiles,\cvshortfiles}
\edef\allexceptcvandcvlongfiles{\resumefiles,\cvshortspecialfiles}
%\usepackage[utf8x]{inputenc}
%\usepackage[T1]{fontenc}
%\usepackage{helvetica} % uses helvetica postscript font (download helvetica.sty)
%\usepackage{newcent}   % uses new century schoolbook postscript font
\usepackage[T1]{fontenc}
\usepackage[margin=1in,top=0.5in,bottom=0.5in]{geometry}
\usepackage{hyperref}
\usepackage{calc}
\usepackage{amsmath}
\usepackage{etoolbox}[2011/01/03]
%\bibliographystyle{plainyr-rev}
\usepackage[backend=biber,maxbibnames=99,sorting=ymdnt,noerroretextools=true,style=alphabetic]{biblatex}
% http://tex.stackexchange.com/a/140641
\DeclareSortingTemplate{ymdnt}{
    \sort{
        \field{presort}
    }
    \sort[final]{
        \field{sortkey}
    }
    \sort[direction=descending]{
        \field{sortyear}
        \field{year}
        \literal{9999}
    }
    \sort[direction=descending]{
        \field[padside=left,padwidth=2,padchar=0]{month}
        \literal{99}
    }
    \sort[direction=descending]{
        \field[padside=left,padwidth=2,padchar=0]{day}
        \literal{99}
    }
    \sort{
        \field{sortname}
        \field{author}
        \field{editor}
        \field{translator}
        \field{sorttitle}
        \field{title}
    }
    \sort{
        \field{sorttitle}
        \field{title}
    }
}
\addbibresource{jason-gross.bib}
\DeclareBibliographyCategory{exclude}
\makeatletter
\@ifpackageloaded{etextools}{%
}{%
  \let\origifempty\ifempty
  \let\origforlistloop\forlistloop
  \undef\ifempty
  \usepackage{etextools}
  \let\eifempty\ifempty
  \let\ifempty\origifempty
  \let\forlistloop\origforlistloop
  \def\preetextools{\let\ifempty\eifempty}
}%
\makeatother
\begin{format}
  \employer{l}\dates{r}\\
  \title{l}\location{r} \\
  \body
\end{format}
\let\oldemployer=\employer
\let\oldtitle=\title
\renewcommand{\employer}[1]{\oldemployer{\textbf{#1}}}
\renewcommand{\title}[1]{\oldtitle{\textit{#1}}}

% \let\oldposition=\position
% \let\oldendposition=\endposition
% \renewenvironment{position}{\oldposition\vspace*{-2\baselineskip}}{\endoldposition}

\makeatletter
\let\old@itemize=\itemize
\def\itemize{\old@itemize\setlength{\itemsep}{0pt}\setlength{\parskip}{0pt}\setlength{\leftskip}{-1em}}
\renewenvironment{position}
  {%
  \begingroup
    \par
      \the\tabular@head
%     \addpenalty{-\@secpenalty}% bad place for a page break
    \penalty -\@secpenalty % bad place for a page break
    \penalty 10000
    \ignorespaces
    \vspace*{-\baselineskip}
  }{%
      \the\tabular@tail
%     \addpenalty{\@secpenalty}% good place for a page break
    \penalty \@secpenalty % good place for a page break
    \endgroup
}
\makeatother

\newsectionwidth{1em}
\resumewidth=7.5in

\newcommand{\aswidthof}[2]{\rlap{#1}\hphantom{#2}}

\name{\LARGE \bf \parbox{\widthof{github.com/JasonGross}}{%
    \centering Jason Gross \\
    \vspace{0.2cm}
    \large \href{https://github.com/JasonGross/}{github.com/JasonGross} \\
    \href{https://people.csail.mit.edu/jgross/}{people.csail.mit.edu/jgross} \\
    $\left.\right.$ \\
    $\left.\right.$ \\%
}}

\vspace{5cm}

%\address{}%\textsc{Address}\\258 Prospect Street, Apt \# 1L\\Cambridge, MA 02139}
%\address{\textsc{Contact}\\\href{mailto:jgross@mit.edu}{jgross@mit.edu}\\(631) 790-8962}
\newcommand{\addressaswidth}[1]{\makebox[\widthof{\textsc{jgross@mit.edu}}][c]{#1}}
\address{\addressaswidth{\textsc{Contact}}\\\addressaswidth{\href{mailto:jgross@mit.edu}{jgross@mit.edu}}\\\addressaswidth{(631) 790-8962}}
\end{multitex}

\begin{document}

\begin{multitex}
\resume

\section{\textsc{Research Interests}}
\begin{itemize}
\item Programming Languages and AI: Löb's Theorem, Type Theory, Compact Proof Generation
\item Trust and Security Engineering: Formal Verification, Cryptography, Performance Engineering
\end{itemize}

\section{\textsc{Experience}}

\employer{Machine Intelligence Research Institute (MIRI)}
\dates{February 2021--Present}
\title{Research Staff}
\location{Berkeley, CA (remote)}
\begin{position}
  \begin{itemize}
  \item Performing self-directed research into topics in fundamental programming language theory and mathematics%, bringing insights and understanding back to my supervisor
  \end{itemize}
\end{position}

\employer{Coq Development Team, INRIA}
\dates{June 2021--Present}
\title{Core Team Member}
\location{Nantes, France (remote)}
\begin{position}
  \begin{itemize}
  \item Reported the plurality of all-time bugs in Coq (since 2012)
  \item Designed and engineered a bug minimizer for automatically producing minimized stand-alone test-cases from bug reports with automatic support for minimizing regressions in external projects tested by Coq CI
  \item Researching performance issues that impact scalability of automated verification
  \end{itemize}
\end{position}

\employer{MIT CSAIL}
\dates{September 2013--February 2021}
\title{PhD Researcher}
\location{Cambridge, MA}
\begin{position}
  \begin{itemize}
  \item Main Project: Fiat Cryptography (\href{https://github.com/mit-plv/fiat-crypto}{github.com/mit-plv/fiat-crypto})
  \item Fiat Cryptography is a developer tool to generate proven-correct cryptographic code. The project has wide industry adoption, including powering secure connections in Google Chrome and Firefox
  \item Lead development of one of the world's first algorithm-level-optimizing compilers
  \item Collaboratively implemented the tool; wrote backends to C, Go, Java, and JSON; managed development of backends to Rust and Zig
  \end{itemize}
\end{position}

\section{\textsc{Education}}
\employer{Massachusetts Institute of Technology}
\dates{2013--2021}
\title{{\rm PhD in Computer Science}}
\location{Cambridge, MA}
\begin{position}\\
  Advisor: Adam Chlipala\\
  \textit{Thesis: Performance Engineering of Proof-Based Software Systems at Scale}\\
  \textit{SM Thesis: An Extensible Framework for Synthesizing Efficient, Verified Parsers}
\end{position}

\employer{Massachusetts Institute of Technology}
\dates{2009--2013}
\title{{\rm BS in Mathematics and Physics}}
\location{Cambridge, MA}
\begin{position}\\
  GPA: 4.6/5
\end{position}

\clearpage

\section{\textsc{Internships}}

\begin{itemize}
\item MIRI, summer 2019: Formalized type theories, and proved properties of programs that reason about themselves
\item Google, summer 2018: Worked on integration of Fiat Cryptography with BoringSSL in Chrome
\item Google, summer 2016: Extended Fiat Cryptography with ECC primiatives for integration with Open Titan
\item Microsoft Research, summer 2014: Collaboratively created a language for specifying input/output behavior of x86 assembly programs, verfied the input/output behavior of a number of simple programs, and improved performance of the x86proved project
\item MIT CSAIL - PLV, 2012 - 2014: Entered a significant amount of category theory into the automated proof assistant Coq, and worked on building an interface for databases and database migration on top of category theory
\item MIT CSAIL - CoCoSci, 2009 - 2011: Designed and managed the data collection webpage for research in categorical learning and transfer learning
\item Commack High School, 2006 - 2009: Researched circuits over sets of natural numbers, winning 4th (2009) and 3rd (2008) place awards in mathematics at ISEF
\end{itemize}

\section{\textsc{Professional Activities}}

\begin{itemize}
\item Co-maintainer of the Fiat Cryptography project
\item Co-maintainer of the homotopy type theory Coq repository (\href{https://github.com/HoTT/HoTT}{\texttt{HoTT/HoTT} on GitHub})
\item Program Committee Member of ITP 2023 and CoqPL 2022
\item Supervising research in formalizing correspondence of affine logic to two-player games
\item Supervising research in anti-inductive utility functions
\item Supervising research in performative power of predidiction markets
\item Circling Facilitator at The Relateful Company
\item Member of SIPB (Student Information and Processing Board)
\end{itemize}

\section{\textsc{Select Past Activities}}

\begin{itemize}
\item Particpant in MIRI Decision Theory Workshops
\item Volunteer at CFAR workshops
\item Instructor at MIT ESP Programs
\item Instructor at Monsoon Math Camp
\item President of MIT Tech Squares
\item Contributor to the SIPB BarnOwn project
\item Project leader for MITeX, an online interface for composing \LaTeX
\item TA for 6.172: Performance Engineering
\item TA for 8.012: Physics I and 8.022: Physics II at the Experimental Study Group
\item Participant at Cananda/USA Mathcamp
\end{itemize}

\section{\textsc{Programming Languages}}
\begin{itemize}
\item
  Proficient: Coq, %\onlymultitex{\cvlongfiles}{\TeX{} macro language, }
  Mathematica, git, Python, JavaScript, BASIC
\item
  Working knowledge: %\onlymultitex{\cvlongfiles}{\LaTeX{}, }
  C, C++, Agda, OCaml, Haskell, Scheme, HTML, CSS, Perl, Java
\item
  Basic knowledge: Matlab, Lean, Idris, Ruby, Go, Ur/Web, x86 Assembly
\end{itemize}

\nocite{*}

\clearpage
\section{\textsc{Selected Presentations and Publications}}
\addtocategory{exclude}{Evaluation2021Huhmann}
\printbibliography[title={$\left.\right.$},notcategory=exclude]

\endresume
\end{multitex}
\end{document}
