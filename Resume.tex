\documentclass[11pt]{res}

\usepackage[T1]{fontenc}
\usepackage[margin=1in,top=0.5in,bottom=0.5in]{geometry}
\usepackage{hyperref}
\usepackage{calc}
\usepackage{amsmath}
\usepackage{etoolbox}[2011/01/03]
\expandafter\def\csname blx@noerroretextools\endcsname{}
\usepackage[backend=biber,maxbibnames=99,sorting=ymdnt,style=alphabetic]{biblatex}
% http://tex.stackexchange.com/a/140641
\DeclareSortingTemplate{ymdnt}{
    \sort{
        \field{presort}
    }
    \sort[final]{
        \field{sortkey}
    }
    \sort[direction=descending]{
        \field{sortyear}
        \field{year}
        \literal{9999}
    }
    \sort[direction=descending]{
        \field[padside=left,padwidth=2,padchar=0]{month}
        \literal{99}
    }
    \sort[direction=descending]{
        \field[padside=left,padwidth=2,padchar=0]{day}
        \literal{99}
    }
    \sort{
        \field{sortname}
        \field{author}
        \field{editor}
        \field{translator}
        \field{sorttitle}
        \field{title}
    }
    \sort{
        \field{sorttitle}
        \field{title}
    }
}
\addbibresource{jason-gross.bib}
\DeclareBibliographyCategory{exclude}
\makeatletter
\@ifpackageloaded{etextools}{%
}{%
  \let\origifempty\ifempty
  \let\origforlistloop\forlistloop
  \undef\ifempty
  \usepackage{etextools}
  \let\eifempty\ifempty
  \let\ifempty\origifempty
  \let\forlistloop\origforlistloop
  \def\preetextools{\let\ifempty\eifempty}
}%
\makeatother
\begin{format}
  \employer{l}\dates{r}\\
  \title{l}\location{r} \\
  \body
\end{format}
\let\oldemployer=\employer
\let\oldtitle=\title
\renewcommand{\employer}[1]{\oldemployer{\textbf{#1}}}
\renewcommand{\title}[1]{\oldtitle{\textit{#1}}}

% \let\oldposition=\position
% \let\oldendposition=\endposition
% \renewenvironment{position}{\oldposition\vspace*{-2\baselineskip}}{\endoldposition}

\makeatletter
\let\old@itemize=\itemize
\def\itemize{\old@itemize\setlength{\itemsep}{0pt}\setlength{\parskip}{0pt}\setlength{\leftskip}{-1em}}
\renewenvironment{position}
  {%
  \begingroup
    \par
      \the\tabular@head
%     \addpenalty{-\@secpenalty}% bad place for a page break
    \penalty -\@secpenalty % bad place for a page break
    \penalty 10000
    \ignorespaces
    \vspace*{-\baselineskip}
  }{%
      \the\tabular@tail
%     \addpenalty{\@secpenalty}% good place for a page break
    \penalty \@secpenalty % good place for a page break
    \endgroup
}
\makeatother

\newsectionwidth{1em}
\makeatletter
\def\firstemployerskip{\vskip-\baselineskip\def\employerskip{}}
\pretocmd{\employer}{\employerskip}{}{}
\apptocmd{\@@section}{\vskip1.5\baselineskip\relax\def\employerskip{\firstemployerskip}}{}{}
\makeatother
\resumewidth=7.5in

\newcommand{\aswidthof}[2]{\rlap{#1}\hphantom{#2}}

\name{\LARGE Jason Gross\vspace{0.75cm}}
%\parbox{\textwidth}{%\widthof{github.com/JasonGross}}{%
%    {\LARGE \centering \textbf{Jason Gross}} \\
%    \vspace{0.2cm}
%    \large
%    \href{https://github.com/JasonGross/}{github.com/JasonGross} \\
%    \href{https://people.csail.mit.edu/jgross/}{people.csail.mit.edu/jgross} \\
%    $\left.\right.$ \\
%    $\left.\right.$ \\%
%}}



\address{\parbox{\textwidth}{%
  \href{mailto:jgross@mit.edu}{\large Email: \normalsize jgross@mit.edu}\hfill
  \href{https://people.csail.mit.edu/jgross/}{\large Website: \normalsize people.csail.mit.edu/jgross} \\
  \href{https://github.com/JasonGross/}{\large GitHub: \normalsize JasonGross}\hfill
  \href{https://scholar.google.com/citations?user=QouPlrMAAAAJ&hl=en}{\large Google Scholar: \normalsize QouPlrMAAAAJ}%
}%
}
%\address{}%\textsc{Address}\\258 Prospect Street, Apt \# 1L\\Cambridge, MA 02139}
%\address{\textsc{Contact}\\\href{mailto:jgross@mit.edu}{jgross@mit.edu}\\(631) 790-8962}
%\newcommand{\addressaswidth}[1]{\makebox[\widthof{\textsc{jgross@mit.edu}}][c]{#1}}
%\address{\addressaswidth{\textsc{Contact}}\\\addressaswidth{\href{mailto:jgross@mit.edu}{jgross@mit.edu}}\\\addressaswidth{(631) 790-8962}}

\begin{document}

\resume

%\section{\textsc{Research Interests}}
%\begin{itemize}
%\item AI: Mechanistic Interpretability, Heuristic Arguments, Compact Proofs, Löb's Theorem
%\item Security \& Engineering: Formal Verification, Cryptography, Bugs \& CI Infrastructure
%\item Programming Languages: Type Theory, Performance, Proof Engines \& Automation
%%
%%\item Mechanistic Interpretability: Heuristic Arguments, Compact Proofs, Type Theory
%%\item Security Engineering: Formal Verification, Cryptography, Performance Engineering
%%\item Programming Languages and AI: Löb's Theorem, Type Theory, Compact Proof Generation
%%%\item Trust and Security Engineering: Formal Verification, Cryptography, Performance Engineering
%\end{itemize}
\section{\textsc{About Me}}
\vspace*{-0.25cm}
Research scientist with extensive experience in proofs and formal verification transitioning to AI \& ML.
Currently researching mechanistic interpretability, heuristic arguments, and compact proofs.
Programming language polyglot, with working knowledge of around two dozen programming languages, and expertise in a handful (Coq (1M+ loc), Python (40k+ loc), Agda (40k+ loc), others).
I spent the years of my PhD on low-level cryptographic code generation, automation, performance engineering, and infrastructure around debugging and CI.


% TODO: https://github.com/avgupta456/github-trends
\section{\textsc{Experience}}

\employer{Special Project of ARC Theory}
\dates{August 2023--Present}
\title{Project Lead}
\location{Berkeley, CA}
\begin{position}
  \begin{itemize}
  \item Building the first machine-checked functional-correctness proofs of mechanistic interpretability arguments about transformers in Coq (\href{https://github.com/JasonGross/neural-net-coq-interp}{\texttt{github.com/JasonGross/neural-net-coq-interp}})
  \item Raised \$150k and leading an interdisciplinary team of ML researchers and mathematicians
  \end{itemize}
\end{position}

\employer{\mbox{Machine Intelligence Research Institute (MIRI)}}
\dates{February 2021--Present}
\title{Research Staff}
\location{Berkeley, CA (remote)}
\begin{position}
  \begin{itemize}
  \item Performing self-directed research into fundamental programming language theory and math%ematics%, bringing insights and understanding back to my supervisor
  \end{itemize}
\end{position}

\employer{Coq Development Team, INRIA}
\dates{June 2021--Present}
\title{Member of Core Team}
\location{Nantes, France (remote)}
\begin{position}
  \begin{itemize}
  \item Reported the plurality of all-time bugs in Coq (since 2012)
  \item Designed and engineered a bug report minimizer for automatically producing minimized standalone test-cases and minimizing regressions in external projects tested by Coq's CI
  \item Researching performance issues that impact scalability of automated verification
  \end{itemize}
\end{position}

\employer{MIT CSAIL}
\dates{September 2013--February 2021}
\title{PhD Researcher}
\location{Cambridge, MA}
\begin{position}
  \begin{itemize}
  \item Main Project: Fiat Cryptography (\href{https://github.com/mit-plv/fiat-crypto}{\texttt{github.com/mit-plv/fiat-crypto}})
  \item Fiat Cryptography is a developer tool to generate proven-correct cryptographic code, with wide industry adoption, powering the plurality of secure connections in Chrome and Firefox
  \item Lead development of one of the world's first algorithm-level-optimizing compilers
  \item Collaboratively implemented the tool; wrote backends to C, Go, Java, and JSON; managed development of backends to Rust and Zig
  \end{itemize}
\end{position}

\section{\textsc{Education}}
\employer{Massachusetts Institute of Technology}
\dates{2013--2021}
\title{{\rm PhD in Computer Science}}
\location{Cambridge, MA}
\begin{position}\\
  Advisor: Adam Chlipala\\
  \textit{Thesis: Performance Engineering of Proof-Based Software Systems at Scale}\\
  \textit{SM Thesis: An Extensible Framework for Synthesizing Efficient, Verified Parsers}
\end{position}

\employer{Massachusetts Institute of Technology}
\dates{2009--2013}
\title{{\rm BS in Mathematics and Physics}}
\location{Cambridge, MA}
\begin{position}\\
  GPA: 4.6/5
\end{position}

\clearpage

\section{\textsc{Internships}}

\begin{itemize}
\item MIRI, summer 2019: Formalized type theories, and proved properties of programs that reason about themselves
\item Google, summer 2018: Worked on integration of Fiat Cryptography with BoringSSL in Chrome
\item Google, summer 2016: Extended Fiat Cryptography with ECC primiatives for integration with Open Titan
\item Microsoft Research, summer 2014: Collaboratively created a language for specifying input/output behavior of x86 assembly programs, verfied the input/output behavior of a number of simple programs, and improved performance of the x86proved project
\item MIT CSAIL --- PLV, 2012--2014: Entered a significant amount of category theory into the automated proof assistant Coq, and worked on building an interface for databases and database migration on top of category theory
\item MIT CSAIL --- CoCoSci, 2009--2011: Designed and managed the data collection webpage for research in categorical learning and transfer learning
\item Commack High School, 2006--2009: Researched circuits over sets of natural numbers, winning 4th (2009) and 3rd (2008) place awards in mathematics at ISEF
\end{itemize}

\section{\textsc{Professional Activities}}

\begin{itemize}
\item Co-maintainer of the Fiat Cryptography project
\item Co-maintainer of the homotopy type theory Coq repository (\href{https://github.com/HoTT/HoTT}{\texttt{HoTT/HoTT} on GitHub})
\item Program Committee Member of ITP 2023 and CoqPL 2022
\item Supervising research in formalizing correspondence of affine logic to two-player games
\item Supervising research in anti-inductive utility functions
\item Supervising research in performative power of predidiction markets
\item Circling Facilitator at The Relateful Company
\item Member of SIPB (Student Information and Processing Board)
\end{itemize}

\section{\textsc{Select Past Activities}}

\begin{itemize}
\item Particpant in MIRI Decision Theory Workshops
\item Volunteer at CFAR workshops
\item Instructor at MIT ESP Programs
\item Instructor at Monsoon Math Camp
\item President of MIT Tech Squares
\item Contributor to the SIPB BarnOwn project
\item Project leader for MITeX, an online interface for composing \LaTeX
\item TA for 6.172: Performance Engineering
\item TA for 8.012: Physics I and 8.022: Physics II at the Experimental Study Group
\item Participant at Cananda/USA Mathcamp
\end{itemize}

\section{\textsc{Programming Languages}}
\begin{itemize}
\item
  Proficient: Coq, Agda, Python, %\onlymultitex{\cvlongfiles}{\TeX{} macro language, }
  Mathematica, BASIC, \TeX{} macro language, git, JavaScript
\item
  Working knowledge: %\onlymultitex{\cvlongfiles}{\LaTeX{}, }
  C, C++, OCaml, Haskell, Scheme%/Racket
  , HTML, CSS, Perl, Bash, Java
\item
  Basic knowledge: MATLAB, Lean, Idris, Ruby, Go, Ur/Web, x86 Assembly, SQL, Batch
\end{itemize}

\nocite{*}

\clearpage
\section{\textsc{Selected Presentations and Publications}}
\addtocategory{exclude}{Evaluation2021Huhmann}
\printbibliography[title={$\left.\right.$},notcategory=exclude]

\endresume
\end{document}
